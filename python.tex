\documentclass[11pt, oneside]{article}   	% use "amsart" instead of "article" for AMSLaTeX format
\usepackage{geometry}                		% See geometry.pdf to learn the layout options. There are lots.
\geometry{letterpaper}                   		% ... or a4paper or a5paper or ... 
%\geometry{landscape}                		% Activate for rotated page geometry
%\usepackage[parfill]{parskip}    		% Activate to begin paragraphs with an empty line rather than an indent
\usepackage{graphicx}				% Use pdf, png, jpg, or eps§ with pdflatex; use eps in DVI mode
								% TeX will automatically convert eps --> pdf in pdflatex		
\usepackage{amssymb}

%SetFonts

%SetFonts


\title{Python}
\author{Akihiro Minamino}
%\date{}							% Activate to display a given date or no date

\begin{document}
\maketitle
\section{変数}
Python変数の重要なポイントは、変数はただの名前だということである。
データをいれているオブジェクトに名前を付けるだけである。
名前は値自体ではなく値の参照である。
名前は、オブジェクトに貼るポストイットのようなものである。\\
 \\
Pythonの数字の並びは、リテラル\footnote{リテラルとは、プログラムのソースコードにおいて使用される、数値や文字列を直接に記述した定数のことである。変数の対義語であり、変更されないことを前提とした値である。}の整数と見なされる。\\
 \\
Pythonでは、=記号の右辺の式がまず計算され、次に左辺の変数に代入が行われる。
 \\



\end{document}  