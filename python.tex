\documentclass[11pt, oneside]{article}   	% use "amsart" instead of "article" for AMSLaTeX format
\usepackage{geometry}                		% See geometry.pdf to learn the layout options. There are lots.
\usepackage{listings, jlisting}
\renewcommand{\lstlistingname}{リスト}
\lstset{
%language = R,
%language = C++,   
breaklines = true,
numbers = left,
frame = tbrl,
tabsize = 4,
basicstyle =\ttfamily,
captionpos = t
}
\geometry{letterpaper}                   		% ... or a4paper or a5paper or ... 
%\geometry{landscape}                		% Activate for rotated page geometry
%\usepackage[parfill]{parskip}    		% Activate to begin paragraphs with an empty line rather than an indent
\usepackage{graphicx}				% Use pdf, png, jpg, or eps§ with pdflatex; use eps in DVI mode
								% TeX will automatically convert eps --> pdf in pdflatex		
\usepackage{amssymb}

%SetFonts

%SetFonts


\title{Python}
\author{Akihiro Minamino}
%\date{}							% Activate to display a given date or no date

\begin{document}
\maketitle
\section{変数}
\subsection{変数、名前、オブジェクト}
Python変数の重要なポイントは、変数はただの名前だということである。
データをいれているオブジェクトに名前を付けるだけである。
名前は値自体ではなく値の参照である。
名前は、オブジェクトに貼るポストイットのようなものである。

\subsection{数値}
Pythonの数字の並びは、リテラル\footnote{リテラルとは、プログラムのソースコードにおいて使用される、数値や文字列を直接に記述した定数のことである。変数の対義語であり、変更されないことを前提とした値である。}の整数と見なされる。\\
 \\
Pythonでは、=記号の右辺の式がまず計算され、次に左辺の変数に代入が行われる。

\subsection{基数}
整数は、プレフィックスで基数を指定しない限り、10進(基数10)と見なされる。
基数は、「桁上り」しなければならなくなるまで、何個の数字を使えるかを示す。\\
Pythonでは、10進以外に3種類の基数を使ってリテラル整数を表す。
\begin{itemize}
\item 0bは2進(基数2)
\item 0oは8進(基数8)
\item 0xは16進(基数16)
\end{itemize}
インタープリターは、10進整数として、整数を表示する。
\begin{lstlisting}
>>> 10
10
>>> 0b10
2
>>> 0x10
16
\end{lstlisting}

\subsection{型の変換}
Pythonの整数以外のデータ型を整数に変換するには、int()関数を使う。
この関数は整数部だけを残し、小数部を切り捨てる。\\
 \\
int()は、数字でできた文字列を整数に変換する。しかし、小数点や指数部を含む文字列は処理しない。
\begin{lstlisting}
>>> int('98.6')
ValueError: ...
>>> int('1.0e4')
ValueError: ...
\end{lstlisting}

\subsection{文字列}
文字列は、文字のシーケンスである。\\
 \\
 他の言語と異なり、Pythonの文字列はイミュータブルである。
 つまり、文字列をその場で書き換えることができない。\\
  \\
Python文字列は、シングルクォートかダブルクォートで文字を囲んで作る。
どちらのクォートを使っても、Pythonはまったく同じように扱う。
2種類のクォート文字を使えるようにしている理由は、クォート文字を含む文字列を作りやすくするためである。ダブルクォートで文字列を作るときは、文字列内にシングルクォートを入れることができ、シングルクォートで文字列を作るときは、文字列内にダブルクォートを入れることができる。
\begin{lstlisting}
>>> "Nay,' said the naysayer."
"Nay,' said the naysayer."
>>> 'The rare double quote in captivity: ".'
'The rare double quote in captivity: ".'
\end{lstlisting}


\end{document}  